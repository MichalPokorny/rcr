\documentclass[a4paper]{article}
\usepackage{fullpage}
\usepackage[utf8]{inputenc}
% \usepackage{amssymb}
% \usepackage{amsmath}
\usepackage[czech]{babel}
\usepackage{listings}
\usepackage{todonotes}
\usepackage{fancyhdr}
\usepackage{listings}

\def\datum{15. března 2014}

\begin{document}
\title{Uživatelská dokumentace \\ RCR -- OCR knihovna pro Ruby}
\author{Michael Pokorný}
\date{\datum}

\maketitle

\tableofcontents

\section{Základní informace o knihovně}
Účelem knihovny RCR je provádění OCR (Optical Character recognition) nad
předanými obrazovými daty. Jedná se tedy o offline rozpoznávání, neboť
se nepoužívají další informace, jako třeba tlak na pero, pořadí kreslených
čar a podobné. Knihovna dále předpokládá, že vstupní data neobsahují
šum kolem rozpoznávaného písmena. Tento předpoklad se používá k předávání
lepších dat rozpoznávací neuronové síti, avšak to způsobuje problémy,
když se takový šum ve vstupu vyskytne. Je možné knihovnu nastavit tak, aby
tento předpoklad nepoužívala. Takové nastavení však může podstatně snížit
přesnost rozpoznávání.

\section{Konfigurace knihovny}
\todo{napsat}

\section{Rozhraní pro Ruby}
\subsection{Neuronové sítě}
\subsubsection{\texttt{RCR::Data::NeuralNetInput}: vstup pro neuronovou síť}
\subsubsection{\texttt{RCR::Data::Dataset}: abstrakce nad datasety}
\subsubsection{\texttt{RCR::NeuralNet}: neuronové sítě}
\todo{obal nad FANN}

\subsection{\texttt{RCR::Classifier}: klasifikátory}
\subsubsection{\texttt{RCR::Classifier::Neural}: klasifikátory založené na
neuronových sítích}

\subsection{\texttt{RCR::MarkovChain}: Markovovy řetězce}
\subsection{\texttt{RCR::LanguageModel}: jazykové modely}
\subsubsection{\texttt{RCR::LanguageModel::MarkovChain}: jazykové modely
založené na Markovových řetězcích}

\subsection{\texttt{RCR::GUI}: grafické komponenty}
\subsubsection{\texttt{RCR::GUI::LetterDrawingArea}: komponenta pro vstup
písmene}

\section{Rozhraní pro ostatní jazyky}
Ke zjednodušení použití knihovny v jiných jazycích obsahuje balíček několik
samostatně spustitelných programů. Všechny tyto utility používají globální
konfiguraci RCR.

\subsection{\texttt{rcr-classify-letter}}
Tento program je určený ke klasifikaci obrázků obsahujících samostatná písmena.
Tato písmena očekává v libovolném formátu přijímaném knihovnou ImageMagick,
mimo jiné tedy akceptuje široce rozšířené formáty PNG a JPG.

Syntaxe volání:
\begin{lstlisting}
$ rcr-classify-letter [--show-alternatives] (filename1) (filename2) ...
\end{lstlisting}

Parametry \texttt{filenameX} jsou názvy souborů, ze kterých se mají přečíst
obrazová data. Je-li některý z názvů souborů pouze "\texttt{-}", pak se data
pro toto písmeno přečtou ze standartního vstupu. Ze standartního vstupu
není možné číst vstup více než jednou za spuštění.

K předání názvů souborů začínajících pomlčkou je potřeba tyto názvy předat
za standartním Unixovým oddělovačem argumentů a souborů: \texttt{--}.

Parametr \texttt{--show-alternatives} způsobí, že pro vstupní soubory
uvedené za tímto parametrem se na standartní výstup nevypíšou jenom
písmena, kterým přidělil klasifikátor nejvyšší skóre, ale vypíšou se
všechny hypotézy klasifikátoru sestupně podle skóre. Každá hypotéza
pro písmeno je uvedena na samostatné řádce ve formátu \texttt{(písmeno)
(skóre)}. Seznamy hypotéz pro vstupní soubory jsou na výstupu odděleny prázdnou
řádkou.

Příklady použití:
\begin{lstlisting}
$ rcr-classify-letter A.png
A
$ cat C.png | rcr-classify-letter --show-alternatives - D.png
C 0.87
O 0.04
Q 0.04
Z 0.03
I 0.02

D 0.94
O 0.05
Q 0.01

$ cat C.png | rcr-classify-letter A.png B.jpg - -- -D.png -E.png
A
B
C
D
E
\end{lstlisting}

\end{document}
