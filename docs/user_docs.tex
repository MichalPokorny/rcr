% TODO: zvyraznit syntaxi?

\documentclass[a4paper]{article}
\usepackage{fullpage}
\usepackage[utf8]{inputenc}
% \usepackage{amssymb}
% \usepackage{amsmath}
\usepackage[czech]{babel}
\usepackage{listings}
\usepackage{todonotes}
\usepackage{fancyhdr}
\usepackage{listings}

\definecolor{string}{rgb}{0.7,0.0,0.0}
\definecolor{comment}{rgb}{0.13,0.54,0.13}
\definecolor{keyword}{rgb}{0.0,0.0,1.0}
\lstset{
% numbers=left,
  language=Ruby,
  frame=single,
  tabsize=2,
  basicstyle=\footnotesize\ttfamily,
  keywordstyle=\color{keyword}\textbf,
  commentstyle=\color{comment}\textit,
  stringstyle=\color{string}
}
\pgfdeclarelayer{background}
\pgfdeclarelayer{foreground}
\pgfsetlayers{background,main,foreground}

\def\datum{15. března 2014}

\begin{document}
\title{Uživatelská dokumentace \\ RCR -- OCR knihovna pro Ruby}
\author{Michael Pokorný}
\date{\datum}

\maketitle

\tableofcontents

\section{Základní informace o knihovně}
Účelem knihovny RCR je provádění OCR (Optical Character recognition) nad
předanými obrazovými daty. Jedná se tedy o offline rozpoznávání, neboť
se nepoužívají další informace, jako třeba tlak na pero, pořadí kreslených
čar a podobné. Knihovna dále předpokládá, že vstupní data neobsahují
šum kolem rozpoznávaného písmena. Tento předpoklad se používá k předávání
lepších dat rozpoznávací neuronové síti, avšak to způsobuje problémy,
když se takový šum ve vstupu vyskytne. Je možné knihovnu nastavit tak, aby
tento předpoklad nepoužívala. Takové nastavení však může podstatně snížit
přesnost rozpoznávání.

Knihovna RCR používá sémantické verzování.

\section{Instalace}
\todo{pozadavky: ruby 2 atd.}
\todo{napsat}

\section{Konfigurace knihovny}
\todo{napsat}

\section{Rozhraní pro Ruby}

\subsection{Zjednodušené rozhraní}
Aby nebylo potřeba při každém použití knihovny složitě vytvářet klasifikátory,
obsahuje knihovna i zjednodušené rozhraní. Toto rozhraní umožňuje jedním
příkazem vytvořit komponenty knihovny podle základní konfigurace.

Toto rozhraní se importuje příkazem \texttt{require 'rcr/easy'}.
\todo{priklad rychleho pouziti}

\subsection{Neuronové sítě}
\subsubsection{\texttt{RCR::Data::NeuralNetInput}: vstup pro neuronovou síť}
Třída \texttt{RCR::Data::NeuralNetInput} se importuje pomocí \texttt{require
'rcr/data/neural\_net\_input'}. Tato třída se používá všude v knihovně, kde
je se pracuje s daty, která půjdou na vstup neuronové síti. Účelem třídy
je omezit málo restriktivní rozhraní, které by nabízelo použití jiných datových
typů (například pole). Tato třída znesnadňuje svým uživatelům provádět operace,
které by mohly porušit invariant "jedná se o validní vstup neuronové sítě".
Pokud by například chyba v kódu způsobila, že se neuronové síti předá pole
obsahující kromě čísel také položky typu \texttt{String}, mohla by se projevit
velice hluboko v kódu RCR. Takto se tato logická chyba objeví dříve: ve chvíli,
kdy by došlo k pokusu vytvořit \texttt{RCR::Data::NeuralNetInput} z pole,
jehož položky nejsou pouze čísla.

Veřejné rozhraní \texttt{RCR::Data::NeuralNetInput} se skládá z konstruktoru a
metod \texttt{\#==}, \texttt{\#concat}, \texttt{\#size} a \texttt{\#data}.

Konstruktor bere jako jediný parametr \texttt{Array}-like objekt obsahující
prvky typu \texttt{Float}.
\begin{lstlisting}
input = RCR::Data::NeuralNetInput.new([0.1, 0.2, 0.3])
\end{lstlisting}

Metoda \texttt{\#==} podle své standartní sémantiky porovnává dva objekty typu
\texttt{RCR::Data::NeuralNetInput}. Platí zde obvyklé problémy s aritmetikou
v plovoucí čárce.
\begin{lstlisting}
input == RCR::Data::NeuralNetInput.new([0.1, 0.2, 0.3]) # true
\end{lstlisting}

Metoda \texttt{\#concat} spojuje dva vstupy za sebe:
\begin{lstlisting}
input.concat(RCR::Data::NeuralNetInput.new([0.4, 0.5])) ==
  RCR::Data::NeuralNetInput([0.1, 0.2, 0.3, 0.4, 0.5])
\end{lstlisting}

\todo{chci zachovat metodu data???}

\subsubsection{\texttt{RCR::Data::Dataset}: abstrakce nad datasety}
Třída \texttt{RCR::Data::Dataset} se importuje pomocí \texttt{require
'rcr/data/dataset'}. Jedná se o abstrakci nad datasetem: uspořádanou
množinou párů "vstup - očekávaný výstup".

Konstruktor této třídy přijímá buď parametr typu \texttt{Array}, nebo parametr
typu \texttt{Hash}.

Parametr typu \texttt{Array} se interpretuje jako seznam
párů "vstup - očekávaný výstup":
\begin{lstlisting}
# Dataset pro funkci 'y = x0 + x1'
dataset = RCR::Data::Dataset.new([
  [[0.1, 0.1], 0.2], [[0.3, 0.5], 0.8], [[-0.5, 0.5], 0], [[0.1, -0.3], -0.2]
])
\end{lstlisting}

Předá-li se konstruktoru parametr typu \texttt{Hash}, jsou jeho
klíče interpretovány jako očekávané výsledky a hodnoty jako
pole vstupů, které mají dát daný výsledek:
\begin{lstlisting}
# Dataset pro funkci 'y = x0 + x1'
dataset2 = RCR::Data::Dataset.new({
  0.0 => [[-0.1, 0.1], [0.0, 0.0], [0.5, -0.5]],
  0.5 => [[-0.1, 0.6], [0.4, 0.1], [0.2, 0.3]],
  -0.5 => [[-0.2, -0.3], [0.2, -0.7], [-0.9, 0.4]],
})
\end{lstlisting}

Třída implementuje funkce \texttt{\#empty?}, \texttt{\#shuffle!} a
\texttt{\#size} obdobným způsobem jako standartní třída \texttt{Array}
(tedy dotaz na prázdnost, náhodné přeházení na místě a dotaz na počet párů
vstup-výstup).

Funkce \texttt{\#each} iteruje přes všechny páry klíč-hodnota:
\begin{lstlisting}
# [-0.1, 0.1] -> 0.0, [0.0, 0.0] -> 0.0, ...
text = ""
dataset2.each { |pair| text << "#{pair.first.inspect} -> #{pair.last}, " }
puts text
\end{lstlisting}

Funkce \texttt{\#split} s volitelným pojmenovaným parametrem \texttt{threshold}
rozdělí instanci na dva kusy v poměru daném tímto parametrem. Jeho standartní
hodnota je \texttt{0.8}. Tato hodnota způsobí, že první vrácený poddataset bude
přibližně $4\times$ větší než druhý.
Tato funkce se dá použít například na pohodlné rozdělení na trénovací a
testovací data:
\begin{lstlisting}
train, test = dataset2.split(threshold: 0.75)
\end{lstlisting}

Metoda \texttt{\#restrict\_keys} je určena pro odstranění nechtěných klíčů
z datasetu. Dá se použít například máme-li dataset obsahující správné klasifikace
obrázků písmen a čísel, ale chceme-li trénovat pouze klasifikátor na čísla.
Obsah původního datasetu zůstane nezměněný, dataset s méně klíči se pouze vrátí.
\begin{lstlisting}
number_dataset = whole_dataset.restrict_keys('0'..'9')
\end{lstlisting}

\todo{zbytek API}

\subsubsection{\texttt{RCR::NeuralNet}: neuronové sítě}
\todo{obal nad FANN}

\subsection{\texttt{RCR::Classifier}: klasifikátory}
\subsubsection{\texttt{RCR::Classifier::Neural}: klasifikátory založené na
neuronových sítích}

\subsection{\texttt{RCR::MarkovChain}: Markovské řetězce}
\subsection{\texttt{RCR::LanguageModel}: jazykové modely}
\subsubsection{\texttt{RCR::LanguageModel::MarkovChain}: jazykové modely
založené na Markovských řetězcích}

\subsection{\texttt{RCR::GUI}: grafické komponenty}
\subsubsection{\texttt{RCR::GUI::LetterDrawingArea}: komponenta pro vstup
písmene}

\section{Rozhraní pro ostatní jazyky}
Ke zjednodušení použití knihovny v jiných jazycích obsahuje balíček několik
samostatně spustitelných programů. Všechny tyto utility používají globální
konfiguraci RCR.

\subsection{\texttt{rcr-classify-letter}}
Tento program je určený ke klasifikaci obrázků obsahujících samostatná písmena.
Tato písmena očekává v libovolném formátu přijímaném knihovnou ImageMagick,
mimo jiné tedy akceptuje široce rozšířené formáty PNG a JPG.

Syntaxe volání:
\begin{lstlisting}
$ rcr-classify-letter [--show-alternatives] (filename1) (filename2) ...
\end{lstlisting}

Parametry \texttt{filenameX} jsou názvy souborů, ze kterých se mají přečíst
obrazová data. Je-li některý z názvů souborů pouze "\texttt{-}", pak se data
pro toto písmeno přečtou ze standartního vstupu. Ze standartního vstupu
není možné číst vstup více než jednou za spuštění.

K předání názvů souborů začínajících pomlčkou je potřeba tyto názvy předat
za standartním Unixovým oddělovačem argumentů a souborů: \texttt{--}.

Parametr \texttt{--show-alternatives} způsobí, že pro vstupní soubory
uvedené za tímto parametrem se na standartní výstup nevypíšou jenom
písmena, kterým přidělil klasifikátor nejvyšší skóre, ale vypíšou se
všechny hypotézy klasifikátoru sestupně podle skóre. Každá hypotéza
pro písmeno je uvedena na samostatné řádce ve formátu \texttt{(písmeno)
(skóre)}. Seznamy hypotéz pro vstupní soubory jsou na výstupu odděleny prázdnou
řádkou.

Příklady použití:
\begin{lstlisting}
$ rcr-classify-letter A.png
A
$ cat C.png | rcr-classify-letter --show-alternatives - D.png
C 0.87
O 0.04
Q 0.04
Z 0.03
I 0.02

D 0.94
O 0.05
Q 0.01

$ cat C.png | rcr-classify-letter A.png B.jpg - -- -D.png -E.png
A
B
C
D
E
\end{lstlisting}

\section{Příklady}
Adresář \texttt{examples/} obsahuje několik praktických příkladů použití knihovny.

\subsection{Tester klasifikátoru písmen
(\texttt{examples/letter-classifier-gui})}


\end{document}
