% TODO: train letter classifieru chce dataset z inputu, ktere jsou image
% TODO: zvyraznit syntaxi?

\documentclass[a4paper]{article}
\usepackage{fullpage}
\usepackage[utf8]{inputenc}
% \usepackage{amssymb}
% \usepackage{amsmath}
\usepackage[czech]{babel}
\usepackage{listings}
\usepackage{todonotes}
\usepackage{fancyhdr}
\usepackage{listings}

\definecolor{string}{rgb}{0.7,0.0,0.0}
\definecolor{comment}{rgb}{0.13,0.54,0.13}
\definecolor{keyword}{rgb}{0.0,0.0,1.0}
\lstset{
% numbers=left,
  language=Ruby,
  frame=single,
  tabsize=2,
  basicstyle=\footnotesize\ttfamily,
  keywordstyle=\color{keyword}\textbf,
  commentstyle=\color{comment}\textit,
  stringstyle=\color{string}
}
\pgfdeclarelayer{background}
\pgfdeclarelayer{foreground}
\pgfsetlayers{background,main,foreground}

\def\datum{15. března 2014}
\def\githuburl{git clone https://github.com/MichalPokorny/rcr.git}

\begin{document}
\title{Uživatelská dokumentace \\ RCR -- OCR knihovna pro Ruby}
\author{Michael Pokorný}
\date{\datum}

\maketitle

\tableofcontents

\section{Základní informace o knihovně}
Účelem knihovny RCR je provádění OCR (Optical Character recognition) nad
předanými obrazovými daty. Jedná se tedy o offline rozpoznávání, neboť
se nepoužívají další informace, jako třeba tlak na pero, pořadí kreslených
čar a podobné. Knihovna dále předpokládá, že vstupní data neobsahují
šum kolem rozpoznávaného písmena. Tento předpoklad se používá k předávání
lepších dat rozpoznávací neuronové síti, avšak to způsobuje problémy,
když se takový šum ve vstupu vyskytne. Je možné knihovnu nastavit tak, aby
tento předpoklad nepoužívala. Takové nastavení však může podstatně snížit
přesnost rozpoznávání.

Knihovna RCR používá sémantické verzování.

Vývoj probíhá přes verzovací systém Git. Hlavní repozitář se nachází
na adrese \texttt{\githuburl}.

Používání tohoto softwaru je povoleno pod MIT licencí (viz \texttt{LICENSE.txt}).

\section{Instalace a infrastruktura}
\subsection{Požadavky}
Kromě níže popsaných Ruby gemů (softwarových balíků) požaduje knihovna RCR
nainstalované Ruby verze aspoň 2, grafický toolkit GTK+ 2 a ImageMagick.
Knihovna byla vytvářena s použitím verzovacího systému Git.

\subsubsection{Ruby gemy}
\begin{itemize}
\item \texttt{chunky\_png}, \texttt{oily\_png}: knihovny pro práci s PNG soubory
	a nativní varianty některých jejich metod (jako optimalizace).
	% TODO: preformulovat?
\item \texttt{ruby-fann}: Ruby rozhraní k C knihovně FANN, která implementuje
	základní neuronové sítě. Obsahuje vlastní kopii této knihovny.
\item \texttt{rmagick}: Ruby rozhraní k velmi rozšířené sadě nástrojů pro
	manipulaci s grafikou ImageMagick.
\item \texttt{gtk2}: Ruby bindingy pro GTK+ 2. Používají se v grafických
	komponentách, které knihovna implementuje.
% TODO: rubyfish??? chci ho?
\item \texttt{rubyfish}: Ruby port Python knihovny \texttt{jellyfish}, která
	provádí přibližné a fonetické porovnávání řetězců.
\end{itemize}

Následující gemy jsou vyžadovány pouze pro vývoj knihovny nebo pro balíčkování
ze zdrojových kódů.
\begin{itemize}
\item \texttt{bundler}: správce závislostí Ruby gemů
\item \texttt{rake}: Ruby obdoba GNU Make
\end{itemize}

\subsection{Instalace z RubyGems}
Knihovna RCR je Ruby gem.
\todo{napsat postup instalace a zverejnit to}

\subsection{Lokální balíčkování a instalace}
Druhá možnost, jak knihovnu nainstalovat, je vytvořit balíček ze zdrojových
souborů v repozitáři. Lokální kopii repozitáře lze vytvořit spuštěním
\texttt{\githuburl}.

Po nainstalování gemů \texttt{rake} a \texttt{bundler} a spuštění
\texttt{bundle install} se do systému nainstalují potřebné balíky. Poté
stačí spustit \texttt{rake build} v libovolném podadresáři knihovny
k zabalíčkování do adresáře \texttt{pkg}. Příkaz \texttt{rake install} pak
umožňuje tento balík nainstalovat.

\begin{lstlisting}
~ $ git clone https://github.com/MichalPokorny/rcr.git
Cloning into 'rcr'...
remote: Counting objects: 1167, done.
remote: Compressing objects: 100% (460/460), done.
remote: Total 1167 (delta 621), reused 1144 (delta 602)
Receiving objects: 100% (1167/1167), 190.61 KiB | 270.00 KiB/s, done.
Resolving deltas: 100% (621/621), done.
Checking connectivity... done.

~ $ cd rcr

~/rcr $ gem install rake bundler
Fetching: rake-10.1.1.gem (100%)
Successfully installed rake-10.1.1
Parsing documentation for rake-10.1.1
Installing ri documentation for rake-10.1.1
Done installing documentation for rake after 5 seconds
Fetching: bundler-1.5.3.gem (100%)
Successfully installed bundler-1.5.3
Parsing documentation for bundler-1.5.3
Installing ri documentation for bundler-1.5.3
Done installing documentation for bundler after 11 seconds
2 gems installed

~/rcr $ bundle install
\end{lstlisting}
\todo{otestovat ze to jde!!! (tady to v labu selhalo)}
\todo{vypnout ruby syntaxi}

\subsection{Automatické testy}
Knihovna obsahuje sadu automatických testů používajících modul
\texttt{Test::Unit} ze standartní knihovny jazyka Ruby.
Testy se nachází v adresáři \texttt{test} a spouštějí se příkazem
\texttt{rake} nebo \texttt{rake test}. Některá dodatečná data potřebná
ke spuštění testů se nachází v adresáři \texttt{test-data}.

\section{Konfigurace knihovny}
\todo{napsat}

\section{Rozhraní pro Ruby}

\subsection{Zjednodušené rozhraní}
Aby nebylo potřeba při každém použití knihovny složitě vytvářet klasifikátory,
obsahuje knihovna i zjednodušené rozhraní. Toto rozhraní umožňuje jednoduše
používat obvyklé funkce a jedním příkazem vytvořit komponenty knihovny podle
základní konfigurace.

Toto rozhraní se importuje příkazem \texttt{require 'rcr/easy'}.

\subsubsection{Načítání obrazu}
K jednoduchému načítání obrazových dat k použití knihovnou RCR slouží metody
\texttt{RCR.load\_image} a \texttt{RCR.load\_image\_from\_blob}. Metoda
\texttt{RCR.load\_image} bere jako jediný parametr cestu k souboru k načtení.
Metoda \texttt{RCR.load\_image\_from\_blob} oproti tomu bere jako svůj parametr
obsah tohoto souboru.
\begin{lstlisting}
image = RCR.load_image("letter.png")
image2 = RCR.load_image_from_blob(File.read("letter.png"))
\end{lstlisting}

\subsubsection{Vytváření komponent dle standartní konfigurace}
Metody \texttt{RCR.build\_letter\_classifier} a
\texttt{RCR.build\_language\_model} nahrají klasifikátor písmen, respektive jazykový model
dle standartní konfigurace. Těmto metodám je možné předat jako argument
\texttt{Hash}, kterým lze tuto konfiguraci upravit.
\begin{lstlisting}
classifier = RCR.build_letter_classifier(letter_classifier_path: "/tmp/classifier")
model = RCR.build_language_model
\end{lstlisting}

Díky těmto metodám je jednoduché vyrobit a používat komponenty knihovny
bez hluboké znalosti jejich funkcí:
\begin{lstlisting}
classifier = RCR.build_letter_classifier
image = RCR.load_image("letter.png")
puts "Detected letter: #{classifier.classify(image)}"
\end{lstlisting}

\subsection{Práce s obrazem}
\todo{napsat}
\subsubsection{\texttt{RCR::Data::Image}: základní reprezentace obrazu}
\todo{napsat}
\subsubsection{\texttt{RCR::Data::Imagelike}: líná reprezentace obrazu}
\todo{napsat}

\paragraph{\texttt{RCR::Data::MergedImagelike}}
\todo{napsat}
\paragraph{\texttt{RCR::Data::CroppedImagelike}}
\todo{napsat}
\paragraph{\texttt{RCR::Data::CairoImagelike}}
\todo{napsat}
\paragraph{\texttt{RCR::Data::PixmapImagelike}}
\todo{napsat}

\subsection{Neuronové sítě}
\subsubsection{\texttt{RCR::Data::NeuralNetInput}: vstup pro neuronovou síť}
Třída \texttt{RCR::Data::NeuralNetInput} se importuje pomocí \texttt{require
'rcr/data/neural\_net\_input'}. Tato třída se používá všude v knihovně, kde
je se pracuje s daty, která půjdou na vstup neuronové síti. Účelem třídy
je omezit málo restriktivní rozhraní, které by nabízelo použití jiných datových
typů (například pole). Tato třída znesnadňuje svým uživatelům provádět operace,
které by mohly porušit invariant "jedná se o validní vstup neuronové sítě".
Pokud by například chyba v kódu způsobila, že se neuronové síti předá pole
obsahující kromě čísel také položky typu \texttt{String}, mohla by se projevit
velice hluboko v kódu RCR. Takto se tato logická chyba objeví dříve: ve chvíli,
kdy by došlo k pokusu vytvořit \texttt{RCR::Data::NeuralNetInput} z pole,
jehož položky nejsou pouze čísla.

Veřejné rozhraní \texttt{RCR::Data::NeuralNetInput} se skládá z konstruktoru a
metod \texttt{\#==}, \texttt{NeuralNetInput.concat}, \texttt{\#size} a \texttt{\#data}.

Konstruktor bere jako jediný parametr \texttt{Array}-like objekt obsahující
prvky typu \texttt{Float}. Vložení jiných datových typů (včetně jiných číselných
typů jako \texttt{BigDecimal} nebo \texttt{Integer}) je chyba.
% NeuralNetInputTest#test_basic
\begin{lstlisting}
input = RCR::Data::NeuralNetInput.new([0.1, 0.2, 0.3])
\end{lstlisting}

Metoda \texttt{\#==} podle své standartní sémantiky porovnává dva objekty typu
\texttt{RCR::Data::NeuralNetInput}. Platí zde obvyklé problémy s aritmetikou
v plovoucí čárce.
% NeuralNetInputTest#test_basic
\begin{lstlisting}
input == RCR::Data::NeuralNetInput.new([0.1, 0.2, 0.3]) # true
\end{lstlisting}

Metoda \texttt{NeuralNetInput.concat} spojuje dva nebo více vstupů za sebe:
% NeuralNetInputTest#test_basic
\begin{lstlisting}
NeuralNetInput.concat(input, RCR::Data::NeuralNetInput.new([0.4, 0.5])) ==
  RCR::Data::NeuralNetInput([0.1, 0.2, 0.3, 0.4, 0.5])
\end{lstlisting}

\todo{chci zachovat metodu data???}

\subsubsection{\texttt{RCR::Data::Dataset}: abstrakce nad datasety}
Třída \texttt{RCR::Data::Dataset} se importuje pomocí \texttt{require
'rcr/data/dataset'}. Jedná se o abstrakci nad datasetem: uspořádanou
množinou párů "vstup - očekávaný výstup". Tato třída se nepoužívá jen
při práci s neuronovými sítěmi, ale i například při práci s klasifikátory.

Konstruktor této třídy přijímá buď parametr typu \texttt{Array}, nebo parametr
typu \texttt{Hash}.

Parametr typu \texttt{Array} se interpretuje jako seznam
párů "vstup - očekávaný výstup":
% RCR::Data::DatasetTest#test_basic
\begin{lstlisting}
# Dataset pro funkci 'y = x0 + x1'
dataset = RCR::Data::Dataset.new([
  [[0.1, 0.1], 0.2], [[0.3, 0.5], 0.8], [[-0.5, 0.5], 0.0], [[0.1, -0.3], -0.2]
])
\end{lstlisting}

Tato třída kontroluje konzistenci datových typů vstupů a očekávaných výstupů.
K zjištění uloženého typu vstupů, respektive výstupů je možné použít metody
\texttt{#input\_type} a \texttt{#expected\_output\_type}.
% RCR::Data::DatasetTest#test_basic
\begin{lstlisting}
dataset.input_type == Array && dataset.expected_output_type == Float # true
\end{lstlisting}

Předá-li se konstruktoru parametr typu \texttt{Hash}, jsou jeho
klíče interpretovány jako očekávané výsledky a hodnoty jako
pole vstupů, které mají dát daný výsledek:
% RCR::Data::DatasetTest#test_basic
\begin{lstlisting}
# Dataset pro funkci 'y = x0 + x1'
dataset2 = RCR::Data::Dataset.new({
  0.0 => [[-0.1, 0.1], [0.0, 0.0], [0.5, -0.5]],
  0.5 => [[-0.1, 0.6], [0.4, 0.1], [0.2, 0.3]],
  -0.5 => [[-0.2, -0.3], [0.2, -0.7], [-0.9, 0.4]],
})
\end{lstlisting}

Třída implementuje funkce \texttt{\#empty?}, \texttt{\#shuffle!} a
\texttt{\#size} obdobným způsobem jako standartní třída \texttt{Array}
(tedy dotaz na prázdnost, náhodné přeházení na místě a dotaz na počet párů
vstup-výstup).

Funkce \texttt{\#each} iteruje přes všechny páry klíč-hodnota:
% TODO: test
\begin{lstlisting}
# [-0.1, 0.1] -> 0.0, [0.0, 0.0] -> 0.0, ...
text = ""
dataset2.each { |pair| text << "#{pair.first.inspect} -> #{pair.last}, " }
puts text
\end{lstlisting}

Pomocí metody \texttt{#insert} lze vkládat nová data:
% TODO: test
\begin{lstlisting}
dataset.insert([0.7, 0.2], 0.9)
\end{lstlisting}

Funkce \texttt{\#split} s volitelným pojmenovaným parametrem \texttt{threshold}
rozdělí instanci na dva kusy v poměru daném tímto parametrem. Jeho standartní
hodnota je \texttt{0.8}. Tato hodnota způsobí, že první vrácený poddataset bude
přibližně $4\times$ větší než druhý.
Tato funkce se dá použít například na pohodlné rozdělení na trénovací a
testovací data:
% TODO: test
\begin{lstlisting}
train, test = dataset2.split(threshold: 0.75)
\end{lstlisting}

Metoda \texttt{\#restrict\_expected\_outputs} je určena pro odstranění nechtěných klíčů
z datasetu. Dá se použít například máme-li dataset obsahující správné klasifikace
obrázků písmen a čísel, ale chceme-li trénovat pouze klasifikátor na čísla.
Obsah původního datasetu zůstane nezměněný, dataset s méně klíči se pouze vrátí.
\begin{lstlisting}
number_dataset = whole_dataset.restrict_expected_outputs('0'..'9')
\end{lstlisting}

Pomocí metody \texttt{\#to\_xs\_ys\_arrays} lze z datasetu získat dvojici polí,
kde první z nich obsahuje vstupy a druhé obsahuje na stejných indexech příslušné
očekávané výstupy. Tato dvojice polí se hodí například pro předávání dat do
knihovny FANN.

Metody \texttt{\#transform\_inputs} a \texttt{\#transform\_expected\_outputs} transformují
přes předaný blok všechny vstupy, respektive očekávané výstupy, a vrátí změněný
objekt \texttt{Dataset}. Původní objekt zůstane nezměněný.
% DatasetTest#test_transformations
\begin{lstlisting}
dataset = RCR::Data::Dataset.new([[1, 5], [2, 10], [3, 15]])
dataset2 = dataset.transform_inputs { |x| x * 5 }.
  transform_expected_outputs { |y| y / 5 }
dataset2 == RCR::Data::Dataset.new([[5, 1], [10, 2], [15, 3]]) # true
\end{lstlisting}

Metoda \texttt{\#to\_fann\_dataset} vrací analogickou třídu typu
\texttt{RubyFann::TrainData}. Tato metoda je použitelná pouze jsou-li vstupy i
očekávané výstupy pole čísel. Výsledek lze předat k použití knihovně RubyFann.

\subsubsection{\texttt{RCR::NeuralNet}: neuronové sítě}
Knihovna RCR používá jednoduché feed-forward neuronové sítě. Samotná
implementace algoritmů se nachází v knihovně FANN (Fast Artificial Neural % TODO: reference
Network Library). RCR také používá RubyFann, což je rozhraní nad touto knihovnou
pro jazyk Ruby (samotná knihovna FANN je v jazyce C). Kvůli menším závislostem
na FANN a kvůli méně pohodlnému návrhu rozhraní RubyFann byla vytvořena další
vrstva abstrakce.

Příslušná třída se jmenuje \texttt{RCR::NeuralNet} a je možné ji naimportovat
pomocí \texttt{require 'rcr/neural\_net'}.

Konstruktor této třídy pouze obalí existující objekt reprezentující neuronovou
síť v RubyFann:
\begin{lstlisting}
fann_net = RubyFann::Standard.new(num_inputs: 10, num_outputs: 5,
  hidden_neurons: [8, 8, 8])
net = RCR::NeuralNet.new(fann_net, 10, 5)
\end{lstlisting}

Zavolání \texttt{NeuralNet.create} je zkratka za vytvoření RubyFann neuronové
sítě a její obalení:
\begin{lstlisting}
net = RCR::NeuralNet.create(num_inputs: 10, num_outputs: 5,
  hidden_neurons: [8, 8, 8])
\end{lstlisting}

Metoda \texttt{train} natrénuje neuronovou síť na každém vzoru uloženém
v předaném datasetu. Dataset musí mít typ vstupů \texttt{RCR::Data::NeuralNetInput} a
výstupy musí být pole čísel. Tato metoda provede pouze jedno kolo tréningu.
K získání použitelných výsledků je potřeba volání této metody opakovat.
% NeuralNetTest#test_train_xor
\begin{lstlisting}
# dataset pro funkci XOR
dataset = RCR::Data::Dataset.new([[[0, 0], 0], [[0, 1], 1], [[1, 0], 1], [[1, 1], 0]]).
  transform_inputs { |x| RCR::Data::NeuralNetInput.new(x.map(&:to_f)) }.
  transform_expected_outputs { |y| [y.to_f] }
net = RCR::NeuralNet.create(num_inputs: 2, num_outputs: 1,
  hidden_neurons: [4, 4])
1000.times { net.train(dataset) }
\end{lstlisting}

Nejdůležitější metoda třídy \texttt{RCR::NeuralNet} se jmenuje \texttt{\#run}.
Tato metoda předá neuronové síti vstup typu \texttt{RCR::Data::NeuralNetInput} a
vrátí pole výstupů neuronů na poslední vrstvě.
% NeuralNetTest#test_train_xor
\begin{lstlisting}
output = net.run(RCR::Data::NeuralNetInput.new([0.9, 0.8]))
output[0] < 0.2 # true

output = net.run(RCR::Data::NeuralNetInput.new([0.1, 0.9]))
output[0] > 0.8 # true
\end{lstlisting}

\todo{cascade\_train}

\todo{image transformery}
\todo{marshal}

\subsection{\texttt{RCR::Classifier}: klasifikátory}
\todo{napsat}
\subsubsection{\texttt{RCR::Classifier::Neural}: klasifikátory založené na
neuronových sítích}
\todo{napsat}

\subsection{\texttt{RCR::MarkovChain}: Markovské řetězce}
Knihovna RCR používá Markovské řetězce jako primitiva uvnitř jazykových modelů.
Příslušná třída se nazývá \texttt{RCR::MarkovChain} a importuje se příkazem
\texttt{require 'rcr/markov\_chain'}. Ačkoliv tento popis mluví o Markovských
řetězcích nad písmeny, je tato třída naprogramována obecně -- může pracovat nad
libovolnými objekty.

Markovský řetězec je interně reprezentovaný jako \texttt{Hash}, ve kterém klíče
jsou možné kontexty a hodnoty jsou \texttt{Hash}e, jejichž klíče jsou možná
pokračování a hodnoty jsou jejich pravděpodobnosti. Součet pravděpodobností
pokračování musí být 1.

Konstruktor Markovského řetězce přijímá jako první parametr svou hloubku (0
znamená triviální Markovský řetězec držící jenom relativní četnosti písmen, 1
znamená počítání pravděpodobnosti podle jednoho předcházejícího písmena, atd.).
Druhý volitelný parametr je interní \texttt{Hash}. Je-li uveden, bude tento
řetězec inicializován do příslušného stavu.
% TODO: test
\begin{lstlisting}
mc = MarkovChain.new(2)
mc2 = MarkovChain.new(2, { "AB" => { "C" => 0.7, "A" => 0.3 }, ... })
\end{lstlisting}

Metoda \texttt{\#train} zahodí obsah Markovského řetězce a natrénuje jej na
daném textu.
% TODO: test
\begin{lstlisting}
mc.train("ABC ABD ABA ABC ABC")
\end{lstlisting}

Metoda \texttt{\#score} vrací relativní skóre (pravděpodobnost) daného
pokračování dle kontextu, případně \texttt{nil} pokud tento Markovský řetězec
na toto pokračování nebyl natrénován. První parametr je kontext a druhý parametr
je pokračování k ohodnocení.
% TODO: test
\begin{lstlisting}
mc.score("AB", "C") # 0.6
\end{lstlisting}

\todo{marshal + jde to ted marshalovat?}

\subsection{\texttt{RCR::LanguageModel}: jazykové modely}
Jazykové modely jsou komponenty, které umí klasifikátoru "radit"
reálné šance, že jeho hypotézy jsou správné: text,
který by mohl být po písmenech přečtený jako \texttt{T0PA2} (s číslem
\texttt{0}), může být vhodným modelem opraven na \texttt{TOPAZ}, protože
model může vědět, že je nepravděpodobné, aby se v jednom slově mixovaly
číslice a písmena (nebo může například udržovat množinu slov, která
přijímá).

\todo{definice spolecneho API}

\subsubsection{\texttt{RCR::LanguageModel::MarkovChain}: jazykové modely
založené na Markovských řetězcích}
\todo{napsat}

\subsection{\texttt{RCR::GUI}: grafické komponenty}
\todo{napsat}
\subsubsection{\texttt{RCR::GUI::LetterDrawingArea}: komponenta pro vstup
písmene}
\todo{napsat}

\section{Rozhraní pro ostatní jazyky}
Ke zjednodušení použití knihovny v jiných jazycích obsahuje balíček několik
samostatně spustitelných programů. Všechny tyto utility používají globální
konfiguraci RCR.

\subsection{\texttt{rcr-classify-letter}}
Tento program je určený ke klasifikaci obrázků obsahujících samostatná písmena.
Tato písmena očekává v libovolném formátu přijímaném knihovnou ImageMagick,
mimo jiné tedy akceptuje široce rozšířené formáty PNG a JPG.

Syntaxe volání:
\begin{lstlisting}
$ rcr-classify-letter [--show-alternatives] (filename1) (filename2) ...
\end{lstlisting}

Parametry \texttt{filenameX} jsou názvy souborů, ze kterých se mají přečíst
obrazová data. Je-li některý z názvů souborů pouze "\texttt{-}", pak se data
pro toto písmeno přečtou ze standartního vstupu. Ze standartního vstupu
není možné číst vstup více než jednou za spuštění.

K předání názvů souborů začínajících pomlčkou je potřeba tyto názvy předat
za standartním Unixovým oddělovačem argumentů a souborů: \texttt{--}.

Parametr \texttt{--show-alternatives} způsobí, že pro vstupní soubory
uvedené za tímto parametrem se na standartní výstup nevypíšou jenom
písmena, kterým přidělil klasifikátor nejvyšší skóre, ale vypíšou se
všechny hypotézy klasifikátoru sestupně podle skóre. Každá hypotéza
pro písmeno je uvedena na samostatné řádce ve formátu \texttt{(písmeno)
(skóre)}. Seznamy hypotéz pro vstupní soubory jsou na výstupu odděleny prázdnou
řádkou.

Příklady použití:
\begin{lstlisting}
$ rcr-classify-letter A.png
A
$ cat C.png | rcr-classify-letter --show-alternatives - D.png
C 0.87
O 0.04
Q 0.04
Z 0.03
I 0.02

D 0.94
O 0.05
Q 0.01

$ cat C.png | rcr-classify-letter A.png B.jpg - -- -D.png -E.png
A
B
C
D
E
\end{lstlisting}

\section{Příklady}
Adresář \texttt{examples/} obsahuje několik praktických příkladů použití knihovny.

\subsection{Tester klasifikátoru písmen (\texttt{examples/letter-classifier-gui})}
\todo{sepsat}

\subsection{Hodnocení klasifikátoru písmen (\texttt{examples/evaluate-letter-classifier})}
\todo{sepsat}

\subsection{Jednoduché vyplňování formulářů (\texttt{examples/form-filling})}
\todo{sepsat}

\end{document}
